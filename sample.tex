\documentclass[11pt, a4paper]{article}

\usepackage{amsmath, amssymb, amsfonts} % For math formulas
\usepackage{graphicx}                     % For including plots
\usepackage{geometry}                     % For margins
\usepackage{booktabs}                     % For professional tables
\usepackage{multirow}                     % For merging rows in tables
\usepackage{hyperref}

\geometry{margin=1in}

\title{\textbf{Local Distance-Based Functional Regression: \\ Derivation and Comparative Analysis}}
\author{Research Report}
\date{\today}

\begin{document}

\maketitle

\section{The Model}
We model the scalar response $y$ based on functional predictors $\mathbf{q}$ using a non-parametric kernel regression framework (Nadaraya-Watson). The prediction is a weighted average of training responses, where weights decay exponentially with distance.

\subsection{Estimator Definition}
For a query point $\mathbf{q}_{\text{new}}$, the predicted response $\hat{y}_{\text{new}}$ is:
\begin{equation}
    \hat{y}_{\text{new}} = \sum_{i=1}^n w_{i, \text{new}} \cdot y_i
\end{equation}
where $i$ indexes the training set.

\subsection{Kernel Weights}
The weights $w_{i, \text{new}}$ are normalized such that they sum to 1:
\begin{equation}
    w_{i, \text{new}} = \frac{u_{i, \text{new}}}{\sum_{j=1}^n u_{j, \text{new}}}
\end{equation}
The unnormalized kernel value $u_{i, \text{new}}$ is defined by a Gaussian function:
\begin{equation}
    u_{i, \text{new}} = \exp\left( - \frac{d^2_S(\mathbf{q}_i, \mathbf{q}_{\text{new}})}{\tau} \right)
\end{equation}
Here, $\tau$ is the bandwidth parameter, and $d^2_S(\cdot, \cdot)$ is the Shape distance metric.

\section{Optimization and Derivation}
The critical hyperparameter $\tau$ is learned by minimizing the \textbf{Leave-One-Out Cross-Validation (LOOCV)} error. This ensures the model generalizes well to unseen data.

\subsection{Cost Function}
We seek to minimize the sum of squared leave-one-out errors:
\begin{equation}
    J(\tau) = \sum_{k=1}^n (y_k - \hat{y}_k^{(-k)})^2
\end{equation}
where $\hat{y}_k^{(-k)}$ uses all points $i \neq k$ to predict point $k$.

\subsection{Gradient Derivation}
To optimize $\tau$, we compute the gradient $\frac{\partial J}{\partial \tau}$. Applying the chain rule:

\begin{equation}
    \frac{\partial J}{\partial \tau} = \sum_{k=1}^n \frac{\partial J}{\partial \hat{y}_k} \cdot \frac{\partial \hat{y}_k}{\partial \tau}
\end{equation}

\noindent \textbf{Step 1: Derivative of Cost}
\begin{equation}
    \frac{\partial J}{\partial \hat{y}_k} = -2(y_k - \hat{y}_k)
\end{equation}

\noindent \textbf{Step 2: Derivative of Weights}
Using the quotient rule on $w_{i,k} = u_{i,k} / S_k$ (where $S_k = \sum u$), and noting that $\frac{\partial u}{\partial \tau} = u \cdot \frac{d^2_S}{\tau^2}$:
\begin{equation}
    \frac{\partial w_{i,k}}{\partial \tau} = \frac{w_{i,k}}{\tau^2} \left( d_{ik}^2 - \sum_{j \neq k} w_{j,k} d_{jk}^2 \right)
\end{equation}
Let $\bar{D}_k = \sum_{j \neq k} w_{j,k} d_{jk}^2$ be the locally weighted average distance.

\noindent \textbf{Step 3: Final Gradient}
Substituting back, we obtain the analytical gradient:
\begin{equation}
    \frac{\partial J}{\partial \tau} = -2 \sum_{k=1}^n (y_k - \hat{y}_k) \sum_{i \neq k} y_i \left[ \frac{w_{i,k}}{\tau^2} \left( d_{ik}^2 - \bar{D}_k \right) \right]
\end{equation}
This gradient drives the optimization of $\tau$.

\section{Experimental Results}
In this section, we present the results from two experimental setups: the dataset with length as responses and amplitude as responses. We analyze the Cross-validated Test Set Prediction Performance as a function of the standard deviation of the Gaussian errors.

Figure \ref{fig:new_results} displays the performance curves for three methods: SI-ScoSH, Kernel Regression with $L^2$ distance, and Kernel Regression with Shape distance.

\begin{figure}[h!]
    \centering
    \setlength{\tabcolsep}{2pt}
    \begin{tabular}{cc}
        \includegraphics[width=0.45\textwidth]{KernelvsSIscoShLen.eps} &
        \includegraphics[width=0.45\textwidth]{KernelvsSIscoShAmp.eps} \\
        (a) Dataset Length & (b) Dataset Amplitude
    \end{tabular}
    \caption{Cross-validated Test Set Prediction Performance vs. StdDev of Gaussian errors. The plots compare SI-ScoSH, Kernel Regression ($L^2$), and Kernel Regression (Shape Dist).}
    \label{fig:new_results}
\end{figure}

The interesting thing to note in the plots is as follows: at std.dev values less than 2 in the Length dataset and less than 3 in Amplitude one, SI-Scosh line is above the kernel regression with Shape dist line. But after that , SI-ScoSh drastically decreases but K.reg with Shape Dist decreases less drastically. K.reg with $L_2$ distance is way worse than both.
% \section{Experimental Results}
% We evaluate the proposed framework on two distinct examples. For each, we compare the Standard $L^2$ (ScoSh=0) and Shape Aligned (ScoSh=1) metrics.
% 
% \subsection{Visual Analysis}
% Figure \ref{fig:side_by_side} displays the comparative prediction results for both examples side-by-side.
% 
% \begin{figure}[h!]
%     \centering
%     \setlength{\tabcolsep}{2pt} % Reduce gap between columns
%     \begin{tabular}{cc}
%         \includegraphics[width=0.5\textwidth]{LOOCVLen.eps} &
%         \includegraphics[width=0.5\textwidth]{LOOCVAmp.eps} 
%         \end{tabular}
%     \caption{True vs. Predicted Responses. Blue circles = Standard $L^2$, Red crosses = Shape Aligned. The dashed line is $Y=X$.}
%     \label{fig:side_by_side}
% \end{figure}
% 
% \subsection{Summary of Performance}
% Table \ref{tab:combined_results} summarizes the optimal bandwidths and goodness-of-fit metrics ($R^2$ and MSE) for both examples.
% 
% \begin{table}[h!]
%     \centering
%     \renewcommand{\arraystretch}{1.2}
%     \begin{tabular}{llcc}
%         \toprule
%         \textbf{Dataset} & \textbf{Method} & \textbf{Optimal $\tau$} & \textbf{Test $R^2$} \\
%         \midrule
%         \multirow{2}{*}{\textbf{Len}} 
%           & Standard $L^2$ (ScoSh=0) & 12.71 & 0.819 \\
%           & Shape Aligned (ScoSh=1)  & 4.57 & 0.957 \\
%         \midrule
%         \multirow{2}{*}{\textbf{Amp}} 
%           & Standard $L^2$ (ScoSh=0) & 13.59 & 0.83 \\
%           & Shape Aligned (ScoSh=1)  & 5.22 & 0.95 \\
%         \bottomrule
%     \end{tabular}
%     \caption{Comparative performance metrics for Example 1 and Example 2.}
%     \label{tab:combined_results}
% \end{table}

\end{document}
